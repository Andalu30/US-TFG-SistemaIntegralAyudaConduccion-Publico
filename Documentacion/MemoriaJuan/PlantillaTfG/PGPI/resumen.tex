\cdpchapter{Resumen}\label{sec_resumen}
En este trabajo de Fin de Grado se realiza el diseño e implementación de  un sistema integral de ayuda a la conducción basado en sistemas empotrados de bajo coste.

El objetivo principal es conseguir detectar y avisar al conductor ante estados anómalos de forma acústica y/o visual.

De entre los posibles estados que se consideran importantes a detectar se encuentran la salida de carril, la detección de vehículos en el punto muerto, la aparición de peatones en la trayectoria del vehículo y la superación del límite de velocidad indicado por las correspondientes señales de circulación.

Para realizar esta detección se ha desarrollado una aplicación en Python basada en imagen por computador e inteligencia artificial que nos permite obtener información sobre el exterior e interior del vehículo.
