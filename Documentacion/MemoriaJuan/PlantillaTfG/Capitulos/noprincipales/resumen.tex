\cdpchapter{Resumen}\label{sec_resumen}

Los sistemas de ayuda a la conducción son unos de los más importantes en los vehículos de hoy dia. Se han convertido en un punto importante a tener en cuenta tanto para los usuarios finales, que desean adquirir vehículos seguros que mantengan protegidos a sus familias, como para las propias empresas que los crean, puesto que es un paso intermedio necesario para la eventual transformación a la conducción autónoma.

Sin embargo, este tipo de sistemas suelen estar restringidos a modelos de alta gama o se encuentran tras un \textit{paywall} bastante considerable. 

Es por esto que en este proyecto se describe el diseño e implementación de un sistema integral de ayuda a la conducción basado en sistemas empotrados, visión por computador y modelos de inteligencia artificial que puede ser incluido en cualquier vehículo con relativa facilidad.

El objetivo principal es conseguir detectar y avisar al conductor ante estados anómalos de forma acústica y/o visual.

Para cumplir con este objetivo se ha desarrollado una aplicación en Python que nos permite conectarnos a un simulador y obtener información sobre el exterior e interior del vehículo gracias al uso de varios tipos de sensores \textit{hardware} simulados, procesarla y realizar un análisis del estado actual y futuro del vehículo.